\documentclass{article}
\usepackage{graphicx} % Required for inserting images

\title{Collaboration between Mathematicians and Mathematics 
Educators: dialogical inquiry as a methodological tool 
in Mathematics Education research}
\author{Obidur Rahman}
\date{3 October 2023}

\begin{document}

\maketitle

\section{Introduction}
In this paper the researchers provided a methodological approach called dialogical inquiry which is based of Mikhail Bakhtin's dialogic theory and principles of dialogism. This paper also raises questions about the methods traditionally used for collaborative data analysis and their weaknesses. It also provides methodologies to properly apply dialogical inquiry.

\section{Bakhtinian dialogical theory and dialogism}
The Bakhtinian theory gives us methodologies which is widely used in social sciences. The three principles of Mikhail Bakhtin’s theory of dialogism and dialogical inquriy: motivation,power balance and a process for solving disagreements will help improve the process of Mathematics education research. A really important concept of Bakhtinian theory is superaddressee which is a method of using a higher level addressee in every dialogic exchange. 

\section{Collaborative research in Mathematics Education}
By Mathematicians collaborating and using the method of dialogism they can propose various question about topics.They can have proper dialogical exhanges by using the methods of proper steelman arguements. They can use supperaddresse to figure out an answer from a neutral point of view without getting stuck. Dialogism also opens up various other opportunities for dialogical exchanges. These dialogical exchanges helps the Mathematics reasearchers get a better idea about the didatic theoritical discussions and will give Mathematics educators a better idea about Mathematical exposition. Thus having mutual benefits. Whereas in other methods, having an unequal power dynamic can cause researchers to not have a proper dialogical exchange and ends up stalling the research. 

\section{Conclusion}
This paper attempts to proof that by working together and following Mikhail Bakhtin’s theory of dialogism, Mathematicians will have a new and hybrid understanding of topics. This paper clearly clarifies the importance of collaboration and states the methods of Mikhail Bakhtins Dialogical inquiry.
\end{document}
